\documentclass{article}
\usepackage[utf8]{inputenc}

\usepackage{amsmath}
\usepackage{amsfonts}
\usepackage{amssymb}
\usepackage{graphicx}
\usepackage[spanish]{babel} 


\title{Graph Theory .- Coloring}
\author{Manuel Gijón Agudo}
\date{October 2017}

\usepackage{amsmath}
\usepackage{verbatim} % comentarios

\usepackage{fancyhdr}
\usepackage{tikz}   
\usetikzlibrary{arrows,chains,matrix,positioning,scopes,calc,shapes.geometric}

\textwidth 150mm
\oddsidemargin 4.6mm                
\evensidemargin = \oddsidemargin
\textheight 235mm
\topmargin -3mm
\headsep 2ex

% Cabezera:
\pagestyle{fancy}
\lhead{
\small \itshape \sffamily
Graph Theory}

\rhead{
\thepage}

\cfoot{Manuel Gijón Agudo}

\setlength{\parindent}{4em}
\setlength{\parskip}{1em}

\begin{document}

\newtheorem{teo}{Theorem}[chapter] 

\maketitle

\section*{Exercise 13}
\texttt{Prove that a regular graph of odd order satisfies $\chi\' ( \mathcal{G} ) = \Delta (\mathcal{G}) + 1$ (namely, $\mathcal{G}$ is of class two).}
\par

\noindent\textbl{proof:}

\noindent\textit{Step 1.} Every regular graph of odd order is overfull.



\noindent A graph is overfull if:
$$
|E| = m > \Delta(\mathcal{G}) \left\lfloor{\frac{n}{2}}\right\rfloor =  \Delta(\mathcal{G}) \left\lfloor{\frac{|V|}{2}}\right\rfloor
$$
We apply the Handshaking lemma:
$$
\sum_{v \in V} d(v) = 2 |E| = 2 m
$$
We use that in a regular graph the degree of all vertices is the same.
$$
\sum_{v \in V} d(v) = n d(v) = n \Delta (\mathcal{G}) = 2 m \Rightarrow m = \Delta (\mathcal{G})  \frac{n}{2}
$$


\noindent\textit{Step 2.} If $\mathcal{G}$ is overfull then is of Class 2.


\noindent\textbl{proof:}
\noindentWe make a proof by contradiccion.

\noindent The contrary of the fact that every graph like this is in a class is just to find one that is not (so, it is in class 1, what means that $\chi\' ( \mathcal{G} ) = \Delta (\mathcal{G})$.

\noindent Let $\mathcal{G}$ be a graph whit $n$ vertices and $m$ edges. $\mathcal{G}$ is an overall graph ($|E| = m > \left\lfloor{\frac{n}{2}}\right\rfloor$). 

\noindent Then any  $\Delta (\mathcal{G})$-coloring of edges partitions the set of edges
into $\Delta (\mathcal{G})$ independent subsets. But the number of edges in each independent subset can not be larger than $\left\lfloor{\frac{n}{2}}\right\rfloor$, since otherwise two of these edges would be adjacent.

\noindent It follows that $m \leq \Delta (\mathcal{G}) \left\lfloor{\frac{n}{2}}\right\rfloor $ leading us to a contradiction.

\noindent$\blacksquare$




    
\newpage
\section*{Appendix}


\noindent\textbf{Definition.1 }Let $\mathcal{G} = (V,E)$ be a simple graph, a \textit{proper edge-coloring} of it is a map $c : E \rightarrow \{1, ..., k\}$ such that two incident edges recive distinct colors.


\noindent\textbf{Definition.2 }A \textit{regular graph} is a graph that has $d(x) = N(x)  \forall x \in V$.

\noindent\textbf{Observation.1 }In a regular graph we have $\Delta(\mathcal{G}) = \delta (\mathcal{G}) = N(x)$  $ \forall x \in V$


\noindent\textbf{Definition.2 } The \textit{edge-chromatic number} $\chi \' (\mathcal{G})$ of a graph $\mathcal{G}$ is the minimum positive integer $k$ for which $\mathcal{G}$ admits a proper $k$-edge coloring.


\noindent\textbf{Definition.3 } A graph $\mathcal{G}$ is called \textit{overfull} if:
$$
|E| = m > \Delta(\mathcal{G}) \left\lfloor{\frac{n}{2}}\right\rfloor =  \Delta(\mathcal{G}) \left\lfloor{\frac{|V|}{2}}\right\rfloor
$$

\begin{teo} \textbf{Vizing (1964)}:
The edges of every simple undirected graph may be colored using a number of colors no bigger than $\Delta (\mathcal{G}) + 1$, the maximun degree of the graph plus one. The lower bound $\Delta (\mathcal{G}) \le \chi \' (\mathcal{G})$ is trivial.

\noindent So, we can divide the graps into two categories, \textit{class 1} (their have an edge-chromatic numbar equal to $\Delta (\mathcal{G})$) and \textit{class 2} (when it is equal to $\Delta (\mathcal{G}) + 1$).
\end{teo}




\end{document}

# https://math.stackexchange.com/questions/2175511/prove-k-regular-graph-with-odd-numbe#r-of-vertices-has-chig-geq-k1

# http://web.iyte.edu.tr/~tinabeseri/tina-tez.pdf