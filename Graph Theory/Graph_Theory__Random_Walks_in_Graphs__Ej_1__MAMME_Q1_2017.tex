\documentclass{article}
\usepackage[utf8]{inputenc}

\usepackage{amsmath}
\usepackage{amsfonts}
\usepackage{amssymb}
\usepackage{graphicx}
\usepackage[spanish]{babel} 
\usepackage{amsmath}
\usepackage{verbatim} % comentarios
\usepackage{fancyhdr}
\usepackage{tikz}   
\usetikzlibrary{arrows,chains,matrix,positioning,scopes,calc,shapes.geometric}


\title{Graph Theory .- Random Walks in Graphs, Ej 1, MAMME Q1 2017}
\author{Manuel Gijón Agudo}
\date{December 2017}

\textwidth 150mm
\oddsidemargin 4.6mm                
\evensidemargin = \oddsidemargin
\textheight 235mm
\topmargin -3mm
\headsep 2ex

% Cabezera:
\pagestyle{fancy}
\lhead{
\small \itshape \sffamily
Graph Theory}

\rhead{
\thepage}

\cfoot{Manuel Gijón Agudo}

\setlength{\parindent}{4em}
\setlength{\parskip}{1em}


\begin{document}

\maketitle

\section*{Exercise 1}
% ENUNCIADO
\begin{itemize}
    \item (a) Show that the mixing rate of the $n$-cube with a loop at every vertex is $(n-1)/n + 1$ (considering adding loops to every vertex).
    \item (b) What is the number of steps of a random graph in $Q_{4}^{L}$ ($Q_4$ with a loop at every vertex) such that the propability of being at a vertex is $1/16 \pm 10^{-3}$?
\end{itemize}

\noindent
{\color{gray} \rule{\linewidth}{0.5mm} }

\subsection*{(a)}

\noindent Remind the definition of the $n$-cube: $Q_{n} = K_{2} \Box Q_{n-1} = (K_{2} \Box K_{2})^{n}$.


\noindent Let $K_{2}^{L}$ be the graph $K_{2}$ with two extra edges, a loop over each one of the verteces. We can define the $n$-dimensional cube as the graph $Q_{n}^{L} = K_{2}^{L} \Box Q_{n-1}^{L} = (K_{2}^{L} \Box K_{2}^{L})^{n}$. Notice that this graph is a $(n+1)$-regular graph. 


\noindent We need to remember the eigenvalues of the Laplacian matrix of the hypercube (Exercise 10 from the second part of the Spectral Graph Theory chapter, the key points are written down in the appendix of this paper).

\begin{itemize}
    \item Using the definition of $M$, the transition matrix : $M = D^{-1} A = \frac{1}{r} I A$ (because the graph is $r$-regular).
    \item Realise that the first eigenvalue positive in absolute value is $2$, so: $n + 1 - 2 = n - 1$.
    \item The answer is:
    $$
    \boxed{
    \frac{n - 1}{n + 1}
    }
    $$
\end{itemize}


%%%%%%%%%%%%%%%%%%%%%%%%%%%%%%%%%%%%%%%%%%%%%%%%%%%%%%%%%%%%%%%%%%

\noindent
{\color{gray} \rule{\linewidth}{0.5mm} }
\newpage

\subsection*{(b)}

\noindent The graph is $5$-regular, so $d(i) = 5 (\forall i)$

\noindent We know that the stationary distribution in $Q_4$ is $\pi = \overbrace{(1/16, ..., 1/16)}^{16 \text{ times}}$.

\begin{align}
     \nonumber \pi = (d(1)/2m, ..., d(n)/2m) & = (5/2m, ..., 5/2m) \\ \nonumber & = (1/16, ..., 1/16) \\  \Rightarrow \frac{5}{2m} = \frac{1}{16} \Rightarrow  m = \frac{80}{2} = 40\\ \nonumber
\end{align}

\noindent The eigenvalues are (by exercise 10, second part in spectral graph theory): $0, \pm 2, \pm 4, \pm 6, \pm 8, \pm 10, \pm 12, \pm 14, \pm 16$. In this case the second largest eigenvalue is, by (a), $2/3$.

\noindent Remind that $(v_{1}^{T}) = (\sqrt{d(1)/2m}, ..., \sqrt{d(n)/2m})$.

\begin{align}
     (v_l)_i (v_l)_j = \frac{d(i)}{2m} = \frac{5}{80} = \frac{1}{16}
\end{align}


\noindent We are going to use an inequality that appears in the demostration of the theorem 1.4.

\begin{align}
|p_{k}(i, j) - \pi(j) |^{1/k} \leq \lambda \left| \sum_{l = 2}^{n} (v_l)_i (v_l)_j \sqrt{\frac{d(i)}{d(j)}} \right| ^{1/k}    
\end{align}


\noindent We use (1), (2) and the expresion of the eigenvalue.

\begin{align}
    \nonumber \left| \frac{1}{16} \pm 10^{-3} - \frac{1}{16} \right|^{1/k} = \left|10^{-3} \right|^{1/k} & \leq \lambda \left| 15 \frac{1}{16} \right| ^{1/k} \\ \nonumber & \leq \lambda  \\ \nonumber
\end{align}

\begin{align}
    10^{-3} \leq \lambda^k = \left( \frac{2}{3} \right)^k
\end{align}
 
\noindent With $k = 18$ steps the result is equal to $0.0006766395 = 6.766395 * 10^{-4}$.

%%%%%%%%%%%%%%%%%%%%%%%%%%%%%%%%%%%%%%%%%%%%%%%%%%%%%%%%%%%%%%%%%%
\newpage
\section*{Appendix}

\subsection*{(a)}

\noindent Let $K^{L}_{2} = Q^{L}_{1}$ be the graph $K_2$ with a loop at every vertex. Observe that this is a $2$-regular graph.

\noindent For this graph, the adjacency matrix $A^{L}$ is: 
$$
A^{L}_{1} = 
    \begin{pmatrix}
    1 & 1 \\
    1 & 1
    \end{pmatrix}
    = A_{1} + I_{2}
$$ 
\noindent ,where $A_{1}$ is the adjacency matrix of $K_2 = Q_1$ and $I_2$ the identity matrix of dimension $2$.

\noindent In order to realize the structure of the adjacency matrix for any dimension, let's check the matrix for $K^{L}_{2} \Box K^{L}_{2} = Q^{L}_{2}$:

$$
A^{L}_{2} = 
    \begin{pmatrix}
    1 & 1 & 1 & 0\\
    1 & 1 & 0 & 1\\
    1 & 0 & 1 & 1\\
    0 & 1 & 1 & 1\\
    \end{pmatrix}
    = 
    \begin{pmatrix}
    A^{L}_{1} & I_2\\
    I_2 & A^{L}_{1}\\
    \end{pmatrix}
    = A_{2} + I_{2^2}
$$ 

\noindent ,where $A_{2}$ is the adjacency matrix of $Q_2$ and $I_{2^2}$ the identity matrix of dimension $2^2$.

\noindent So, the structure of the adjacency matrix of $A^{L}_{n}$ will be the next one:

$$
A^{L}_{n} = 
    \begin{pmatrix}
    A^{L}_{n-1} & I_{2^{n-1}}\\
    I_{2^{n-1}} & A^{L}_{n-1}\\
    \end{pmatrix}
    = A_{n} + I_{2^n}
$$ 

\noindent ,where $A_{n}$ is the adjacency matrix of $Q_n$ and $I_{2^{n}}$ the identity matrix of dimension $2^n$.


\noindent Now we have to remind that the Laplacian Matrix of a graph is defined as $L = D - A$, and also the main result of the exercise 10 of the second part of the chapter dedicated to Spectral Graph Theory: the serie of eigenvalues for the Laplacian matrix of the hypercube starts with $0, 2, 4, 6, ...$. So the first non-zero eigenvalue is $2$ for all dimensions.

\end{document}





