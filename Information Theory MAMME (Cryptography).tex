\documentclass{article}
\usepackage[utf8]{inputenc}

\usepackage{amsmath}
\usepackage{amsfonts}
\usepackage{amssymb}
\usepackage{graphicx}
\usepackage[spanish]{babel} 

\usepackage{fancyhdr}
\usepackage{tikz}   
\usetikzlibrary{arrows,chains,matrix,positioning,scopes,calc,shapes.geometric}

\textwidth 150mm
\oddsidemargin 4.6mm                
\evensidemargin = \oddsidemargin
\textheight 235mm
\topmargin -3mm
\headsep 2ex

\pagestyle{fancy}
\lhead{
\small \itshape \sffamily
Information Theory}

\rhead{
\thepage}

\cfoot{Manuel Gijón Agudo}

\setlength{\parindent}{4em}
\setlength{\parskip}{1em}


\title{Information Theory \thanks{Chapter 2 in Crypytography class in Master's degree at UPC}}
\author{Manuel Gijón Agudo }
\date{September 2017 - January 2018}


\begin{document}

	\begin{titlepage}
		\maketitle{} \newpage
		
		\tableofcontents
		\newpage
		
		\section{Introduction}
		\noindent\textbf{Definition 	1: Information} the information coveyed by a source is a function $I:S  \rightarrow [ 0, \infty ) $ where $S$ is a \textbf{source} \footnote{A \textbf{source} is a finite set $S$ together with a sequence of random variables $X_i$ whose range is $S$} with the properties:
		
		\begin{itemize}
			\item $I(s_i)$ is a decreasing function of the propability $p_i$, with $I(s_i) = 0$ if $p_i = 1$.
			\item $I(s_i s_j) = I(s_i) + I(s_j)$
		\end{itemize}
		
	\end{titlepage}

\end{document}