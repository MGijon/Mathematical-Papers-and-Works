\documentclass{article}
\usepackage[utf8]{inputenc}

\usepackage{amsmath}
\usepackage{amsfonts}
\usepackage{amssymb}
\usepackage{graphicx}
\usepackage[spanish]{babel} 
\usepackage{yfonts}
\usepackage{fancyhdr}
\usepackage{tikz}   
\usetikzlibrary{arrows,chains,matrix,positioning,scopes,calc,shapes.geometric}

\textwidth 150mm
\oddsidemargin 4.6mm                
\evensidemargin = \oddsidemargin
\textheight 235mm
\topmargin -3mm
\headsep 2ex

\pagestyle{fancy}
\lhead{
\small \itshape \sffamily
Information Theory}

\rhead{
\thepage}

\cfoot{Manuel Gijón Agudo}

\setlength{\parindent}{4em}
\setlength{\parskip}{1em}


\title{Information Theory \thanks{Chapter 2 in Crypytography class in Master's degree at UPC}}
\author{Manuel Gijón Agudo }
\date{September 2017 - January 2018}


\begin{document}

	\begin{titlepage}
		\maketitle{} 
	\end{titlepage}
	
	\newpage
	\tableofcontents
	\newpage
					
					
		\section{Basic knowledge of coding}

		\subsection{Introduction}
\noindent\textbf{Definition 	1:}  a \textbf{source} is a finite set $\mathcal{S}$ together with a set of random variables ($X_1, X_2, ...$) whose range is $\mathcal{S}$.

If $P(X_n = \mathcal{S}_i)$ only depends on $i$ and not on $n$ then we say the source is \textbf{stationary} and if the $X_n$ are independent then it's \textbf{memoryless}.

Insert example here

\noindent\textbf{Definition 	2:} Let $\mathcal{T}$ be a finite set called \textbf{alphabet}. A map $\mathfrak{C}: \mathbb{S} \longrightarrow \Cup_{n \geq 1} T^{n}$ is called a \textbf{code}.

If $|T| = r$ then $\mathfrak{C}$ is a \textbf{$r$-ary code}. 

\noindent A code extends from $\mathbb{S}$ to $T \cup T^2 \cup ...$ to $\mathbb{S} \cup \mathbb{S}^2 \cup ...$  to $T \cup T^2 \cup ...$ in obvious way.

insert example here


\noindent\textbf{Definition 	3:} The \textbf{average word-lenght} of a code $\mathfrak{C}$ is $L(\mathfrak{C}) := \sum_{i = 1}^{n} p_i l_i$ where $l_i$ is the length of the image of the symbol of $\mathbb{S}$, which is emitted with probability $p_i$.

\noindent For now, we write $\mathfrak{C}$ to be the image of $\mathfrak{C}$. 

		\subsection{Uniquely decodeable codes}
	
\noindent\textbf{Definition 	4:}  If for any sequencies $u_1 ... u_n = v_1 ... v_m$ in $\mathfrak{C}$ implies $m = n$ and $u_i = v_i$ for $i = 1, ..., n$ then we say that $\mathfrak{C}$ is \textbf{uniquely decodeable}.

insert example here

insert example here

insert example here
 
 
\noindent Let $\mathfrak{C}_0 = \mathfrak{C}$:

\begin{itemize}
	\item $\mathfrak{C}_n := \{   \omega \in T \cup T^2 \cup... | u\omega = v$ for some $u \in \mathfrak{C}_{n-1}, v \in \mathfrak{C}$ or $ u\omega = v$ for some $u \in \mathfrak{C}, v \in \mathfrak{C}_{n-1}   \}$
	\item $\mathfrak{C}_{\infty} := \bigcup_{k \geq 1} \mathfrak{C}_k$
\end{itemize}



		\section{Introduction}

\noindent\textbf{Definition 	1:} the \textbf{information} coveyed by a source is a function $I:S  \rightarrow [ 0, \infty ) $ where $S$ is a \textbf{source} \footnote{A \textbf{source} is a finite set $S$ together with a sequence of random variables $X_i$ whose range is $S$} with the properties:
		
		\begin{itemize}
			\item $I(s_i)$ is a decreasing function of the propability $p_i$, with $I(s_i) = 0$ if $p_i = 1$.
			\item $I(s_i s_j) = I(s_i) + I(s_j)$, ie.the information geined by two symbols is the sum of the information obtained from each where the source has symbols $s_1, ..., s_q$ emitted with probabilities $p_1, ..., p_q$.
		\end{itemize}
		
\noindent\textbf{Lemma 	1: } $I(s_i) = - \log_{r}{p_i}$ for some $r$.

\noindent\textit{proof:}  Insert proof here

\noindent
{\color{gray} \rule{\linewidth}{0.5mm} }

\noindent \textbf{Definition 	2:} The \textbf{$r$-ary entropy} $H_{r}(S)$ of a source $S$ is the average information coveyed by $S$.
$$
H_{r}(S) := - \sum_{i = 1}^{q} {p_i \log_{r}{p_i}}
$$
, by convenction $x \log_{r}{x}$ evaluated at $0$ is $0$.

Insert five examples

	\section{Properties of the entropy funcion}
	
\noindent \textbf{Theorem  1:}  $H_{r}(S) \leq \log_{r}{q}$ with equality if and only iff $S$ is the source where each symbol is emitted with probability $1/q$.

\noindent\textit{proof:}  Insert proof here

\noindent
{\color{gray} \rule{\linewidth}{0.5mm} }

\noindent \textbf{Theorem  2:}  $H_{r}(S) \leq L(C)$ for unique decodeable code $C$.

\noindent\textit{proof:}  Insert proof here

\noindent
{\color{gray} \rule{\linewidth}{0.5mm} }


	\section{Shannon-Fano Code}

\noindent Let $S$ be the source with symbols $s_i$ and probabilities $p_i$. Let $l_i := \left\lceil \log_{r}{1/p_i} \right\rceil$. 


\noindent Then: $\sum_{i = 1}^{q}{r^{-l_i}} \leq \sum r^{- \log_{r}{1/p_i}} = \sum p_i = 1$

\noindent \textbf{Definition 	3:} by Kraft exists a prefix code with woed length $l_1, l_2, ..., l_1$. This code is called \textbf{Shannon-Fano code}.

Inert example here


\noindent \textbf{Lemma 	2:}  For the Shannon-Fano code $C$: $H_{r}(S) \leq L(C) < H_{r}(S) + 1$.

\noindent\textit{proof:}  Insert proof here

\noindent
{\color{gray} \rule{\linewidth}{0.5mm} }



	\section{Product of sources}
	
\noindent Let $S$ and $T$ be two memoryless sources, $S$ with symbols $s_i$ and probabilities $p_i$ and $T$ with symbols $t_j$ and probabilities $q_j$.

\noindent \textbf{Definition 	4:} The \textbf{product source} $S \times T$ is a source with symbols $s_{i}t_{j}$ and probabilities $p_{i}q_{j}$.

\noindent \textbf{Theorem  3:} $H_{r}(S \times T) = H_{r} (S) + H_{r} (T)$.

\noindent\textit{proof:}  Insert proof here

\noindent
{\color{gray} \rule{\linewidth}{0.5mm} }

\noindent \textbf{Corollary  1:} $H_{r}(S^n) = n H_{r} (S)$. 


\noindent \textbf{Theorem  4: Noiseless Coding} The average word length $L_n$ of an optiml code of $S^n$ satisfies:
$$
\frac{L_n}{n} \longrightarrow H_{r} (S) \, , n \rightarrow \infty
$$ 
  
\noindent\textit{proof:}  Insert proof here

\noindent
{\color{gray} \rule{\linewidth}{0.5mm} }


some examples

	\section{Markov Chains}
	
\noindent \textbf{Definition 	4:} A \textbf{Markov Chain} is a sequency of random variables where $X_{n + 1}$ depends only for $X_{n}$.

$$
P(X_{n + 1} = s_j | X_{n} = s_j ) = p_{i, j}
$$

\noindent This can be represented in a direct graph and also by a matrix $P := (p)_{i, j}$.

\noindent Suppose $u_0$ is the vector which describes the initial distribution, ie. the $i$-th coordinate of $u_0$ is probability we start at $s_i$. Probability of beeing in the $i$-th state after $r$ steps is the $i$-th coordinate of $u_{0} P^{r}$.

\noindent \textbf{Theorem  5:} if $\exists r \in \mathbb{N}$ such that $P^r$ has no zero entries, then $u_{0} P^{r} \longrightarrow u$, as $n \rightarrow \infty$.

\noindent \textbf{Definition 	5:}  This vector $u$ is called the \textbf{stationary distribution}. It is normalised eigenvector of $P^t$ with eigenvalue $1$, ie. $u_j = \sum_i {p_{i, j} u_i}$ and $\sum_{j} {u_j} = 1$.
 	 
\noindent \textbf{Definition 	6:} If $P$ is the matrix of a Markov Chain and $\exists r$ such that $P^r$ has non zero entries then we say that the Markov Chain is \textbf{regular}.

  
  \section{Sources with memory}
  
Suppose $S$ is a Markov Chain source with random variables $X_1, X_2, ...$ such that
$$
P(X_{n + 1} = s_j | X_{n} = s_j ) = p_{i, j}
$$

\noindent \textbf{Definition 	7:} $S$ is \textbf{not memoryless}, but it is stationary.

 \noindent \textbf{Theorem  6:} suppose $S$ is a regular Markov Chain source with stationary distribution $u = (u_1, ..., u_)$. Let $S'$ be the stationary memoryless source with the same source elements as $S$ (where $s_i$ is emmitted with probability $w_i$). Then:
 
 $$
 H_{r} (S) \leq H_{r} (S')
 $$
 
 \noindent\textit{proof:}  Insert proof here

\noindent
{\color{gray} \rule{\linewidth}{0.5mm} }

\end{document}