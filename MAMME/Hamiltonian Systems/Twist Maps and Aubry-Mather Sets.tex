\documentclass{article}
\usepackage[utf8]{inputenc}

\usepackage{amsmath}
\usepackage{amsfonts}
\usepackage{amssymb}
\usepackage{graphicx}
\usepackage{yfonts}
\usepackage{fancyhdr}
\usepackage{tikz}   
\usetikzlibrary{arrows,chains,matrix,positioning,scopes,calc,shapes.geometric}
\textwidth 150mm
\oddsidemargin 4.6mm                
\evensidemargin = \oddsidemargin
\textheight 235mm
\topmargin -3mm
\headsep 2ex

\pagestyle{fancy}
\lhead{
\small \itshape \sffamily
Twist Maps and Aubry-Mather Sets, Hamiltonian Systems, MAMME}

\rhead{
\thepage}

\cfoot{Manuel Gijón Agudo}

\setlength{\parindent}{4em}
\setlength{\parskip}{1em}


\title{Twist Maps and Aubry-Mather Sets}
\author{Manuel Gijón Agudo }
\date{May, 2019}


\begin{document}

	\begin{titlepage}
		\maketitle{} 
	\end{titlepage}
	
	\newpage
	\tableofcontents
	\newpage
	
%% ============ %%
%% INTRODUCTION %%
%% ============ %%

\section{Introduction}

	
	
	Caso 2 dimensional
	
	Se trataría de dar una introducción y comparar con la teoría KAM explicada por Marcel, o sea que él te podrá dar más detalles.
También en el último capítulo de Meyer Offin hay una aproximación variacional. 


		
	Our objetive in this work is give an introduccion to the Aubry-Mathers theory, explain its importance in the context it is on and expose the fundamental resoutls of the topic. In order to arrive to this point, we first present resoults and definitions in the next section.
		
	%	\color{blue}
	%		\noindent \underline{\textbf{\textit{proof:}}}
	%	\color{black}
		
		
	

	
	%\color{blue}
	%	\noindent \underline{\textbf{\textit{reason why:}}}
		
	%	explanation
	%	\par \noindent \rule{\textwidth}{0.4pt}
	%\color{black}
	

%% ================================  %%
%% FIRST DEFINITIONS AND BASIC RESOULTS   %%
%% ================================  %%		

\section{First definitions and basic resoutls}
		
\noindent bla bla bla bla
		
		
%% ============= %%		
%% MAIN RESOULTS  %%
%% ============= %%		
		
\section{Main resoults}
		
\noindent \textbf{Definition}: \label{def-TwistMap} a monotone \textbf{twist map} is an orientation preserving $\mathcal{C}^{1}$-diffeomorphismm $\varphi: S^1 \times [0, 1] \longrightarrow S^{1} \times [0, 1]$ of an annulus which admits a lift $\bar{\varphi} = (f, g): \mathbb{R} \times [0, 1] \longrightarrow \mathbb{R} \times [0, 1]$ whit the following properties:
\begin{itemize}
	\item [(a)]\label{def-TwistMap_a} $\bar{\varphi}$ preserves (Lebesgue) area.
	\item [(b)]\label{def-TwistMap_b}\textbf{Twist condiction}: $D_{2} f > 0$.
	\item [(c)] \label{def-TwistMap_c}$g(\xi, 0) = 0, \ g(\xi, 1) = 1$.
\end{itemize}

\noindent \textbf{Notes}:
	\begin{itemize}
		\item Instead of (a) we could require $det( \bar{\varphi}' ) = 1$.
		\item Condiction (c) means that $\varphi$ does not commute the boundary components.
		\item We are assuming that the translation $T_{ (1, 0) }, T_{ (1, 0) } (\xi, y)  = (\xi + 1, y)$, generates the group of covering transformations. Then $f( \xi + 1, y) = f(\xi, y) + 1, \ g(\xi + 1, y) = g(\xi, y)$.
	\end{itemize}
		
		
% Monotone twist maps turn up as poincaré maps of somen hamiltonian systems with two degrees of freedom

A fundamental property of the monotone twist maps is that it can be globally described by a generating function:

$$
	H: D \rightarrow \mathbb{R}, \ \ \text{where} \ \ D := \{  (\xi, \eta) \in \mathbb{R}^{2} | f(\xi, 0) \leq \eta \leq f(\xi, 1) 
	\}
$$

up to an additive constant, H is uniquely determined by:

\begin{equation}\label{eq1}
	\bar{\varphi} (x_0, y_0) = (x_1, y_1) \Leftrightarrow
	  \begin{cases} 
     	    - D_{1} H  (x_0, y_0) &= y_0  \\
      		D_{2} H  (x_0, y_0) &= y_1 
      \end{cases}
\end{equation}

\noindent To construct $H$ let $a, b: D \rightarrow \mathbb{R}$ be defined by:
$$
	a ( \xi, f(\xi, y)) := y,  \ \ \ \	b( \xi, f(\xi, \eta)) := g(\xi, a( \xi, \eta))
$$

Then, \ref{eq1} is equivalent to:
$$
	dH = - a d \xi + b d \eta
$$


\noindent From $det( \bar{\varphi}' ) = 1$ can be conclued that the 1-form $\omega = -a d \xi + b d \eta$ is closed.

\noindent Since D is simply connexted, there exits $H: D \rightarrow \mathbb{R}$ such that \ref{eq1} is true. H is $\mathcal{C}^{2}$ in the interior of D and $D_1 H, D_2 H$ and $D_2 D_1 H$ extend continuously to D.  

\noindent Moreover, from (b) in the previous definition implies $D_2 D_1 H = -D_2 a \leq \delta$ for some $\delta > 0$. Finally $H( \xi + 1, \eta + 1) = H(\xi, \eta)$, i.e. $\omega$ is even exact on the cylinder D modulo translations $T_{ (j, j) } \in \mathbb{Z}' $: $\omega$ vanishes along the curve $\xi \rightarrow ( \xi, f(\xi, 0))$ from $(0, f(0, 0))$ to $(1, f(0, 0) + 1)$. 

\noindent PENDIENTE!!!!

\noindent Taking \ref{eq1}  as definition we can extend $\bar{\varphi}$ resp. $\varphi$ to $\mathbb{R}^2$ resp. $S^1 \times \mathbb{R}$. According to this equation, we have a variational principle for the orbits of the extend $\bar{\varphi}$. We recall that a sequence $(x_{i})_{i \in \mathbb{Z}}$ is stationary with respect to H if $D_2 H (x_{i - 1}, x_{i}) + D_1 (x_i, x_{i + 1}) = 0 \ \ \forall i \in \mathbb{Z}$.

\noindent \textbf{Resoult 1}: If $(x_{i})_{i \in \mathbb{Z}}$ is stationary with respect to H then $(x_{i}, y_{i})_{i \in \mathbb{Z}}$, $y_i = - D_{1} H(x_i, x_{i + 1} )$, is an orbit of $\bar{\varphi}$, i.e. $\bar{\varphi}(x_i, y_i) = (x_{i + 1}, y_{i + 1})$. Conversely, if $(x_{i}, y_{i})_{i \in \mathbb{Z}}$ is an orbit of $\bar{\varphi}$ then $(x_{i})_{i \in \mathbb{Z}}$  is stationary with respect to H.


\noindent \textbf{Observation}: Since the $(x_i)_{i \in \mathbb{Z}} \in \mathcal{M} (H)$ are stationary with respect to H every statement on minimal trajectories can be interpreted as a statement on certain types of orbits of the extended $\bar{\varphi}$. We can return to the original $\bar{\varphi}: \mathbb{R} \times [0, 1] \rightarrow \mathbb{R} \times [0, 1]$ in the following way: Let $f_0, f_1 \in 	\bar{G}_{+}$ be defined by $f_{0} (\xi) = f(\xi, 0)$, $f_{1} (\xi) = f(\xi, 1)$ and let $\alpha_0, \alpha_1$ be the rotation number of $f_0, f_1$. The interval $[ \alpha_0, \alpha_1 ]$ is called the \textbf{twist interval} of $\bar{\varphi}$. Later we will see that $\alpha_0 < \alpha_1$.


\noindent \textbf{Lemma}: Suppose $x \in \mathcal{M}_{\alpha}$ and $\alpha \in ( \alpha_0, \alpha_1)$, i.e. $\alpha$ is the interior of the twist interval of $\bar{\varphi}$. Then the orbit $(x_i, y_i)_{i \in \mathbb{Z}}$, $y_i = -D_1 H (x_i, x_{i + 1})$, is contained in 
$\mathbb{R} \times (0, 1)$. 


\noindent \textbf{Theorem 1}: For every irrational $\alpha \in ( \alpha_0, \alpha_1)$ there exists a $\varphi$-invariant set $\mathcal{M} \subseteq S^{1} \times (0, 1)$ with the following properties:
	\begin{itemize}
		\item[(a)] $M_{\alpha}$ is the graph of a Lipschitz function $\psi_{\alpha}: A_{\alpha} \rightarrow (0, 1)$ defined on a closet set $A_{\alpha} \subseteq S^{1}$.
		\item[(b)] $\varphi$ has rotation number $\alpha$  mod $(\mathbb{Z})$ on $M_{\alpha}$, i.e. there exits $h \in \mathcal{G}_{+}$ with $\alpha (h) \equiv \alpha$ mod($\mathbb{Z}$) such that $h(A_{\alpha}) = A_{\alpha}$ and
		
		$$
			\varphi ( \xi, \psi_{\alpha} (\xi) ) = ( h (\xi), \psi_{\alpha} (h(\xi) ) ) \ \ \forall \xi \in A_{\alpha} 		
		$$
		
		\item[(c)] The set $\mathcal{M}_{\alpha}^{\text{rec}}$ of recurrent points in $\mathcal{M}_{\alpha}$ projets either to a cantor set in $S^1$ or to all $S^{1}$. In the latter case $\mathcal{M}_{\alpha}$ is a $\varphi$-invariant Lipschitz curve winding once around the annulus $S^{1} \times [0, 1]$ and $\varphi | \mathcal{M}_{\alpha}$ is topologically conjugate to a rotation.
	\end{itemize}
 
\color{blue}
	\noindent \underline{\textbf{\textit{proof:}}} (we prove (a) - (c) for $\bar{\varphi}$ instead of $\varphi$; everything is invariant counting on $(\xi, \eta) \rightarrow (\xi + 1, \eta)$). Considering the previous lemma, we define:
	
	$$
		\bar{\mathcal{M}}_{\alpha} := \{
		(\xi, \eta) \in \mathbb{R} \times (0, 1) | \exists x \in \mathcal{M}_{\alpha} \ \text{ such that } \ \xi = x_0, \eta = - D_1 H(x_0, x_1)
		\}
	$$

\noindent Acording to \ref{eq1} and the resoult 1 the set $\bar{\mathcal{M}}_{\alpha}$ is $\varphi$-invariant. Acording to the resoutls INSERTAR AQUI LAS REFERENCIAS DE MIERDA MUHAHAHAHA the set $\bar{\mathcal{M}}_{\alpha}$ has one-to-one projection onto a closed set $\bar{A}_{\alpha} \subseteq \mathbb{R}$.

\noindent Considering REFERENCIA 3.19 MUHAHAHAHHA there exists a Lipschitz continuous $\bar{h} \in \bar{ \mathcal{G}_{+} }$ of a rotation number $\alpha$ such that $\bar{h} (x_0) = x_1 \ \ \forall x \in \mathcal{M}_{\alpha}$. Hence $\bar{\psi}_{\alpha} (\xi) := - D_{1} H (\xi, \bar{h} ( \xi ) )$ for $\epsilon \in \bar{A}_{\alpha}$ is Lipschitz as well.

\noindent This proves (a) and (b). Now (c) is a consequence of the properties of circle maps.


	\noindent $\spadesuit$
\color{black}
		
		
\noindent \textbf{Theorem 2}: If C is a $\varphi$-invariant curve winding once around $S^{1} \times [0, 1]$ and if $\varphi | C$ has rotation number $\alpha$ then $C \subseteq \mathcal{M}_{\alpha}$ and $C = \mathcal{M}_{\alpha}$ if $\alpha$ is irrational. 
		
		
%\color{blue}
%	\noindent \underline{\textbf{\textit{proof:}}} 		

%	\noindent $\spadesuit$
%\color{black}
		
\noindent \textbf{Lemma}: For every monotone twist map the twist interval has non-empty interior.

\noindent \textbf{Corollary}: For a standard diffeomorphism $\varphi: S^{1} \times \mathbb{R} \rightarrow S^{1} \times \mathbb{R}$ from (b) in the Theorem 1 choose the function s such that $s' (\xi_{0}) < -2$ for some $\xi_0 \in \mathbb{R}$. THen there is no $\varphi$-invariant curve winding around $S^1 \times \mathbb{R}$.


\color{blue}
	\noindent \underline{\textbf{\textit{proof:}}} 		

	\noindent We know from the Theorem 2 that the existence of a $\varphi$-invariant curve winding around $S^{1} \times \mathbb{R}$ implies the existence of an $x \in \mathcal{M} (H)$ with $x_0 = \xi_0$ where $H(\xi, \eta) = \frac{1}{2} ( \xi - \eta )^2 + S(\xi), \ \ S' = s$. Now we have:
		
	$$
		D_{22} H (x_{-1}, x_0, x_1 ) = D_{22} H( x_{-1}, x_0 ) + D_{11} H( x_0, x_1 ) = 2 + s' ( \xi_0 ) < 0	
	$$
	
	\noindent But this contradicts the minimality of the segment $(x_{-1}, x_0, x_1)$.
	
	\noindent $\spadesuit$
\color{black}

\noindent \textbf{Observation}: all this resoults are true (except the Lypschitz continuity in the Theorem 1) are also true if we relax the conditions of first definition (on a monotone twist map $\varphi$) in the following way:

\begin{itemize}
	\item[(a)] It is sufficient that $\varphi$ is a homeomorphism instead of being a $\mathcal{C}^{1}$-diffeomorphism.
	\item[(b)] Instead of $D_{2} f > 0$ it is suffices that $y \rightarrow f(\xi, y)$ is strictly increasing $\forall \xi \in \mathbb{R}$.
\end{itemize}



%% =========== %%
%% CONCLUSIONS %%
%% =========== %%	
		
\section{Conclusions}
		
\cite{R1}





%% =========== %%
%% BIBLIOGRAPHY %%
%% =========== %%
\newpage		
\begin{thebibliography}{X}
\bibitem{R1} bal bla bla bla
\end{thebibliography}
		
		
		
		
		
		
		
		
		
		
		
		

	
	


\end{document}