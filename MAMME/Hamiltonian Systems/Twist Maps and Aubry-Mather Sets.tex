\documentclass{article}
\usepackage[utf8]{inputenc}

\usepackage{amsmath}
\usepackage{amsfonts}
\usepackage{amssymb}
\usepackage{graphicx}
\usepackage{yfonts}
\usepackage{fancyhdr}
\usepackage{tikz}   
\usetikzlibrary{arrows,chains,matrix,positioning,scopes,calc,shapes.geometric}
\textwidth 150mm
\oddsidemargin 4.6mm                
\evensidemargin = \oddsidemargin
\textheight 235mm
\topmargin -3mm
\headsep 2ex

\pagestyle{fancy}
\lhead{
\small \itshape \sffamily
Twist Maps and Aubry-Mather Sets, Hamiltonian Systems, MAMME}

\rhead{
\thepage}

\cfoot{Manuel Gijón Agudo}

\setlength{\parindent}{4em}
\setlength{\parskip}{1em}


\title{Twist Maps and Aubry-Mather Sets}
\author{Manuel Gijón Agudo }
\date{May, 2019}


\begin{document}

	\begin{titlepage}
		\maketitle{} 
	\end{titlepage}
	
	\newpage
	\tableofcontents
	\newpage
	
%% ============ %%
%% INTRODUCTION %%
%% ============ %%

\section{Introduction}

	
	
	Caso 2 dimensional
	
	Se trataría de dar una introducción y comparar con la teoría KAM explicada por Marcel, o sea que él te podrá dar más detalles.
También en el último capítulo de Meyer Offin hay una aproximación variacional. 


		
	Our objetive in this work is give an introduccion to the Aubry-Mathers theory, explain its importance in the context it is on and expose the fundamental resoutls of the topic. In order to arrive to this point, we first present resoults and definitions in the next section.
		
	%	\color{blue}
	%		\noindent \underline{\textbf{\textit{proof:}}}
	%	\color{black}
		
		
	

	
	%\color{blue}
	%	\noindent \underline{\textbf{\textit{reason why:}}}
		
	%	explanation
	%	\par \noindent \rule{\textwidth}{0.4pt}
	%\color{black}
	

%% ================================  %%
%% FIRST DEFINITIONS AND BASIC RESOULTS   %%
%% ================================  %%		

\section{First definitions and resoults}
		
\noindent \textbf{Definition 1}: $\mathbb{R}^{\mathbb{Z}} := \{ x | x: \mathbb{Z} \rightarrow \mathbb{R}  \}$, the space of bi-infinite sequences of real numbers (we consider it along with the product topology). An element $x = (x_{i})_{i \in \mathbb{Z}}$ is sometimes called \textbf{trajectory}.

\noindent \textbf{Definition 2}: \textbf{Convergence} of a sequence $x^n \in \mathbb{R}^{\mathbb{Z}}$ to $x \in \mathbb{R}^{\mathbb{Z}}$ means that $\lim_{n \rightarrow \infty } x_{i}^{n} = x_{i} \ \ (\forall i \in \mathbb{Z})$.
		
\noindent Given a dunction $H: \mathbb{R}^2 \rightarrow \mathbb{R}$ we can extend H to trajectories (or finite segments) by:
	$$
		H (x_j, \cdots, x_k) = \sum_{i = j}^{k - 1} H ( x_i, x_{i + 1} ) 
	$$
	
\noindent \textbf{Definition 3}:
\begin{itemize}
	\item We say that the segment $(x_j, \cdots, x_k)$ is \textbf{minimal} with respect to H if:
	$$
		H ( x_j, \cdots, x_k) = \sum_{i = j}^{k - 1} H ( x_i, x_{i + 1} )  \leq \sum_{i = j}^{k - 1} H ( x_{i}^{*}, x_{i + 1}^{*} ) = H (x_{j}^{*}, \cdots, x_{k}^{*}) \  (   \forall (x_{j}^{*}, \cdots, x_{k}^{*})  \ \text{ : } \ x_j =  x_{j}^{*}, x_k = x_{k}^{*}   )
	$$
	
	\item $x \in \mathbb{R}^{\mathbb{Z}}$ is \textbf{minimal} with respect to H) if every finite segment of x is minimal (with respect to H).
	
	\item We call $\mathcal{M} = \mathcal{M}(H)$ to the \textbf{set of minimal trajectories} (with respect to H), for $x \in \mathbb{R}^{\mathbb{Z}}$.
\end{itemize}
 
\noindent \textbf{Hypotesis} for H (we assume that it is continuous):
\begin{itemize}
	\item[(H1)] \textit{Periodicity condition}: $ \forall (\xi, \eta) \in \mathbb{R}^{2}: \ H( \xi + 1, \eta + 1) = H(\xi, \eta)$.
	\item[(H2)] \textit{Condition at infinity}: $\lim_{|\eta | \rightarrow \infty } H (\xi, \xi + \eta) = \infty$ uniformly in $\xi$.
	\item[(H3)] \textit{Ordering condiction}: If $\xi_{*} < \xi^{*}, \eta_{*} < \eta^{*} \Rightarrow H( \xi_{*}, \eta_{*}) + H( \xi^{*}, \eta^{*}) < H( \xi_{*}, \eta^{*}) + H( \xi^{*}, \eta_{*})$.
	\item[(H4)] \textit{Transversality condiction}: If $(x_{-1}, x_{0}, x_{1}) \neq (x_{-1}^{*}, x_{0}^{*}, x_{1}^{*})$ are minimal and $x_0 = x_{0}^{*}$ then $(x_{-1} - x_{-1}^{*})(x_{1} - x_{-1}^{*}) < 0$.
\end{itemize}  

\noindent \textbf{Definition 4}: If $H \in \mathcal{C}^{2}$ we say that $x \in \mathbb{R}^{\mathbb{Z}}$ is \textbf{stationary} if:

	$$	
		D_2 H( x_{i -1}, x_{i}) + D_1 H( x_{i}, x_{i + 1}) = 0 \ \ (\forall i \in \mathbb{Z}) 
	$$
	
\noindent \textbf{Observation 1}: each $x \in \mathcal{M}(H)$ is stationary with respect to H.

\noindent \textbf{Definition 5}: We say that $\mathbb{R}^{\mathbb{Z}}$ is \textbf{partially ordered} by $x < x^{*}$ if and only if $x_{i} < x_{i}^{*} \ \ (\forall i \in \mathbb{Z})$.


\noindent \textbf{Definition 6}: $x, x^{*} \in \mathbb{R}^{\mathbb{Z}}$ \textbf{cross}:
	\begin{itemize}
		\item[(a)] at $i \in \mathbb{Z}$ if $x_i = x_{i}^{*}$ and $(x_{i - 1} - x^{*}_{i - 1}) (x_{i + 1} - x^{*}_{i + 1}) < 0 $.
		\item[(b)] between i and $i + 1$ if $(x_{i} - x^{*}_{i}) (x_{i + 1} - x^{*}_{i + 1}) < 0 $.
	\end{itemize}

\noindent \textbf{Observation 2}: Acording to the \textit{transversality condiction} (H4), trajectories $x, x^{*} \in \mathcal{M}$ either are cross or are \textbf{comparable} ($x < x^{*}$, $x = x^{*}$ or $x > x^{*}$).  

\noindent \textbf{Definition 7}:  $x, x^{*} \in \mathbb{R}^{\mathbb{Z}}$ are:
	\begin{itemize}
		\item[-] \textbf{$\alpha$-asymptotic} if $\lim_{i \rightarrow - \infty} | x_i - x_{i}^{*} | = 0$.
		\item[-] \textbf{$\omega$-asymptotic} if $\lim_{i \rightarrow  \infty} | x_i - x_{i}^{*} | = 0$.
		\item[-] \textbf{asymptotic} if they are both, $\alpha$-asymptotic and $\omega$-asymptotic.
	\end{itemize}


\noindent There is an action T on the group $\mathbb{Z}^2$ on $\mathbb{R}^{\mathbb{Z}}$ by order-preserving homeomorphisms: if $(a, b) \in \mathbb{Z}^{2}$ and $x \in \mathbb{R}^{\mathbb{Z}}$ then:

	$$
		T_{(a, b)} x = x^{*} \ \ \text{  where  } x^{*}_{i} = x_{i - a} + b
	$$

The action of $T_{(a, b)}$ on $x$ corresponds to translation of graph(x) $\subseteq \mathbb{R}^{2}$ by $(a, b)$.

\noindent \textbf{Definition 8}: $x \in \mathbb{R}^{\mathbb{Z}}$ is periodic with period $(q, p) \in \big( \mathbb{Z} - \{ 0 \} \big) \times \mathbb{Z}$ if $T_{(a, b)} x = x$.

 
\noindent \textbf{Consecuences of the hypotesis}:
	\begin{itemize}
		\item From H1: We have $H (x) = H( T_{(a, b)} x)$ for every segment $(x_j, \cdots, x_k), k > j$ and every $(a, b) \in \mathbb{Z}^{2}$. In particular $T_{(a, b)}$ maps minimal segments to minimal ones and $\mathcal{M}$ onto itself. The continuity of H implies that $\mathcal{M}$ is closed in $\mathbb{R}^{\mathbb{Z}}$.
		   
		\item From H2: It is possible prove that $\forall (\xi, \eta) \in \mathbb{R}^{2}$ and $\forall i < k$ there exits a minimal segment $(x_j, \cdots, x_k)$ with $x_j = \xi, x_k = \eta$. If $(x_j, \cdots, x_k)$ is minimal then si us evert subsegment $(x_l, \cdots, x_m) \ \ (l \geq j, m \leq k)$.
		
		\item From H3 and H4: \textbf{Lemma 1}: Minimal trajectories cross at most once. If $x \in \mathcal{M}$ and $x^{*} \in \mathcal{M}$ conincide at $i \in \mathbb{Z}$ then $x$ and $x^{*}$ cross at $i \in \mathbb{Z}$.
		
		\color{blue}
			\noindent \underline{\textbf{\textit{proof:}}} 		
			
			\begin{itemize}
				\item[-] Second part comes from the transversality condiction (H4).
				\item[-] We focus on the first part. We use a contradiction argument, we assume that $x$ and $x^{*}$ cross between j and $j + 1$ and between $k$ and $k + 1$, $j < k$. The case where one or both of the crossings take place at an integer can be treated similarly. 
				
				We consider the segments $(x_j, x_{j + 1}^{*}, \cdots, x_{k}^{*}, x_{k + 1})$ and $(x_{j}^{*}, x_{j + 1}, \cdots, x_{k}, x_{k + 1}^{*})$. Using the ordering condictions (H3):
				
				\begin{equation*}
					\begin{split}
						H (x_j, x_{j + 1}^{*}, \cdots, x_{k}^{*}, x_{k + 1}) + H (x_{j}^{*}, x_{j + 1}, \cdots, x_{k}, x_{k + 1}^{*}) &= H(x_j, x_{j + 1}^{*} ) + H (x_{j + 1}^{*}, \cdots, x_{k}^{*}) \\
						&+ H ( x_{k}^{*}, x_{k + 1} ) + H (x_{j}^{*}, x_{j + 1})  \\
						&+ H (x_{j + 1}. \cdots, x_k) + H( x_{k}, x_{k + 1}^{*}) \\
						&< H( x_{j}^{*}, x_{j + 1}^{*}, \cdots, x_{k + 1}^{*} ) \\
						&+ H(x_j, x_{j + 1}, \cdots, x_{k + 1})
					\end{split}
				\end{equation*}	
				
				This contradicts the minimality of at least one of the segments $(x_{j}, \cdots, x_{k + 1})$ and $(x_{j}^{*}, \cdots, x_{k + 1}^{*})$.
							 
			\end{itemize}
	
			\noindent $\spadesuit$
		\color{black}
		
		
		\textbf{Corollary}: If $x \in \mathcal{M}$ and $x^{*} \in \mathcal{M}$ are pericodic with the same period then $x$ and $x^{*}$ do not cross. If $x \in \mathcal{M}$ is periodic with minimal period $(q, p)$ then $q$ and $p$ are relatively prime.
		
		\color{blue}
			\noindent \underline{\textbf{\textit{proof:}}} 		

	
			\noindent $\spadesuit$
		\color{black}
		
		
	\end{itemize}

\noindent \textbf{Theorem 1}: $\forall (q, p) \in \big( \mathbb{Z} - \{ 0 \} \big) \times \mathbb{Z} \ \ \exists x \in \mathcal{M}$ periodic with $(q, p)$.

\color{blue}
	\noindent \underline{\textbf{\textit{proof:}}} 		

	
	\noindent $\spadesuit$
\color{black}
		
		
\noindent \textbf{Lemma 2}: Suppose $x \in \mathcal{M}$ and $x^{*} \in \mathcal{M}$ are $\alpha$-asymptotic (resp. $\omega$-asymptotic) and $| x_{i + 1} - x_i |$ is bounded for $i \rightarrow - \infty$ (resp. $i \rightarrow  \infty$). Then $x$ and $x^{*}$ do not cross.
		
\noindent \textbf{Theorem 2}: Suppose $x \in \mathcal{M}$. Then $x$ and $T_{(a, b)} x$ do not cross for any $(a, b) \in \mathbb{Z}^{2}$.

\color{blue}
	\noindent \underline{\textbf{\textit{proof:}}} 		

	
	\noindent $\spadesuit$
\color{black}

\noindent \textbf{Notation 1}: We call $\bar{B}_{x}$ the closure of $B_{x} := \{ T_{(a, b)} x | (a, b) \in \mathbb{Z}^{2}  \} \subset \mathbb{R}^{\mathbb{Z}}$ and $p_i: \mathbb{R}^{\mathbb{Z}} \rightarrow \mathbb{R}$ the projection $x \rightarrow x_i$ (obviousy is continuous, open and order preserving).  

\noindent \textbf{Lemma 3}: Suppose $x \in \mathcal{M}$. Then $\bar{B}_{x}$ is totally ordered. The projection $p_0$ maps $\bar{B}_{x}$  homeomorphically onto a closed subset of $\mathbb{R}$.

\noindent \textbf{Notation 2}:
\begin{itemize}
	\item[-]  Let $G_{+}$ denote the group of orientation-preserving homeomorphisms of the circle $S^{1} = \mathbb{R} / \mathbb{Z}$.
	\item[-] $\bar{G_{+}} = \{ f | f : \mathbb{R} \rightarrow \mathbb{R} \ \text{  continuous, strictly increasing, } \ f(x+ 1) = f(x) + 1 \}$
\end{itemize}

\noindent \textbf{Theorem 3}: For every $x \in \mathcal{M}$ there exists a a circle map $f \in \bar{G_{+}}$ such that $x_{i + 1} = f (x_i) \ \ (\forall i \in \mathbb{Z})$.


\color{blue}
	\noindent \underline{\textbf{\textit{proof:}}} 		

	
	\noindent $\spadesuit$
\color{black}



DEFINIMOS AQWUI LO QUE VIENE A SER EL ROTACIONAL MUHAHAHAHHHA

\noindent \textbf{Theorem 4}: $\forall \alpha \in \mathbb{R}$ the set $\mathcal{M}_{\alpha} := \{ x \in \mathcal{M} | \alpha(x) = \alpha$ is not empty.


\color{blue}
	\noindent \underline{\textbf{\textit{proof:}}} 		

	
	\noindent $\spadesuit$
\color{black}



%\noindent \textbf{Definition 9}: for every $\varphi \in G_{+}$ we define the \textbf{Poincar\'e rotation number} $\alpha ( \varphi ) \in S^{1}$ COMPLETAR MUHAHAHAHAHAHAHAH


%% ============= %%		
%% MAIN RESOULTS  %%
%% ============= %%		
		
\section{Main resoults}
		
\subsection{Structure of the set of minimal trajectories with irrational rotation number}

\noindent \textbf{Definition 9}: $\mathcal{M}_{\alpha}$ can be describe by a single circle map, and in particular, every $\bar{B_{x}} \subseteq \mathcal{M}_{\alpha}$ contains the set $\mathcal{M}_{\alpha}^{\text{rec}}$ of \textbf{recurrent trajectories} in $\mathcal{M}_{\alpha}$, 

	$$
		\mathcal{M}_{\alpha}^{\text{rec}} := \{ x \in \mathcal{M}_{\alpha} | \exists k_i \in \big( \mathbb{Z}^{2} - \{ 0 \}       \big) \ \text{ such that } \ x = \lim_{i \rightarrow \infty } T_{k_i} x \}
	$$
	
	
\noindent \textbf{Theorem 5}: Suppose $\alpha$ is irrational. Then $\mathcal{M}_{\alpha}$ is totally ordered.


\color{blue}
	\noindent \underline{\textbf{\textit{proof:}}} 		

	ESTO ES LARGO DE COJONES! MUHAHAHAHAHA
	
	\noindent $\spadesuit$
\color{black}


\subsection{Structure of the set of minimal trajectories with rational rotation number}


\noindent For rational $\alpha$, say $\alpha = \frac{p}{q}$ with $p$ and $q$ relatively prime. 


\noindent \textbf{Notation 3}: 
	\begin{itemize}
		\item[-] $P_{q, p} := \{ x \in \mathbb{R}^{\mathbb{Z}} | T_{(q, p)} x = x \}$.
		\item[-] $H_{q, p} := H(x_0, \cdots, x_q)$.
	\end{itemize}


\noindent \textbf{Theorem 6}: $\mathcal{M}_{\alpha}^{\text{rec}}$ is non-empty, closed and totally ordered. Every $x \in \mathcal{M}_{\alpha}^{\text{rec}}$ has a minimal period $(q, p)$. If $x \in \mathcal{M}_{\alpha}^{\text{rec}}$ then $x$ is a minimun of $H_{q, p}: P_{q, p} \rightarrow \mathbb{R}$, in particular $H_{q, p}^{\text{min}} \ (\forall x \in \mathcal{M}_{\alpha}^{\text{rec}} )$.


\color{blue}
	\noindent \underline{\textbf{\textit{proof:}}} 		

	ESTO ES LARGO DE COJONES! MUHAHAHAHAHA
	
	\noindent $\spadesuit$
\color{black}


%% =========== %%
%% CONCLUSIONS %%
%% =========== %%	
		
\section{Conclusions}
		
\cite{R1}





%% =========== %%
%% BIBLIOGRAPHY %%
%% =========== %%
\newpage		
\begin{thebibliography}{X}
\bibitem{R1} bal bla bla bla
\end{thebibliography}
		
		
		
		
		
		
		
		
		
		
		
		

	
	


\end{document}






\\
\noindent \textbf{Definition}: \label{def-TwistMap} a monotone \textbf{twist map} is an orientation preserving $\mathcal{C}^{1}$-diffeomorphismm $\varphi: S^1 \times [0, 1] \longrightarrow S^{1} \times [0, 1]$ of an annulus which admits a lift $\bar{\varphi} = (f, g): \mathbb{R} \times [0, 1] \longrightarrow \mathbb{R} \times [0, 1]$ whit the following properties:
\begin{itemize}
	\item [(a)]\label{def-TwistMap_a} $\bar{\varphi}$ preserves (Lebesgue) area.
	\item [(b)]\label{def-TwistMap_b}\textbf{Twist condiction}: $D_{2} f > 0$.
	\item [(c)] \label{def-TwistMap_c}$g(\xi, 0) = 0, \ g(\xi, 1) = 1$.
\end{itemize}

\noindent \textbf{Notes}:
	\begin{itemize}
		\item Instead of (a) we could require $det( \bar{\varphi}' ) = 1$.
		\item Condiction (c) means that $\varphi$ does not commute the boundary components.
		\item We are assuming that the translation $T_{ (1, 0) }, T_{ (1, 0) } (\xi, y)  = (\xi + 1, y)$, generates the group of covering transformations. Then $f( \xi + 1, y) = f(\xi, y) + 1, \ g(\xi + 1, y) = g(\xi, y)$.
	\end{itemize}
		
		
% Monotone twist maps turn up as poincaré maps of somen hamiltonian systems with two degrees of freedom

A fundamental property of the monotone twist maps is that it can be globally described by a generating function:

$$
	H: D \rightarrow \mathbb{R}, \ \ \text{where} \ \ D := \{  (\xi, \eta) \in \mathbb{R}^{2} | f(\xi, 0) \leq \eta \leq f(\xi, 1) 
	\}
$$

up to an additive constant, H is uniquely determined by:

\begin{equation}\label{eq1}
	\bar{\varphi} (x_0, y_0) = (x_1, y_1) \Leftrightarrow
	  \begin{cases} 
     	    - D_{1} H  (x_0, y_0) &= y_0  \\
      		D_{2} H  (x_0, y_0) &= y_1 
      \end{cases}
\end{equation}

\noindent To construct $H$ let $a, b: D \rightarrow \mathbb{R}$ be defined by:
$$
	a ( \xi, f(\xi, y)) := y,  \ \ \ \	b( \xi, f(\xi, \eta)) := g(\xi, a( \xi, \eta))
$$

Then, \ref{eq1} is equivalent to:
$$
	dH = - a d \xi + b d \eta
$$


\noindent From $det( \bar{\varphi}' ) = 1$ can be conclued that the 1-form $\omega = -a d \xi + b d \eta$ is closed.

\noindent Since D is simply connexted, there exits $H: D \rightarrow \mathbb{R}$ such that \ref{eq1} is true. H is $\mathcal{C}^{2}$ in the interior of D and $D_1 H, D_2 H$ and $D_2 D_1 H$ extend continuously to D.  

\noindent Moreover, from (b) in the previous definition implies $D_2 D_1 H = -D_2 a \leq \delta$ for some $\delta > 0$. Finally $H( \xi + 1, \eta + 1) = H(\xi, \eta)$, i.e. $\omega$ is even exact on the cylinder D modulo translations $T_{ (j, j) } \in \mathbb{Z}' $: $\omega$ vanishes along the curve $\xi \rightarrow ( \xi, f(\xi, 0))$ from $(0, f(0, 0))$ to $(1, f(0, 0) + 1)$. 

\noindent PENDIENTE!!!!

\noindent Taking \ref{eq1}  as definition we can extend $\bar{\varphi}$ resp. $\varphi$ to $\mathbb{R}^2$ resp. $S^1 \times \mathbb{R}$. According to this equation, we have a variational principle for the orbits of the extend $\bar{\varphi}$. We recall that a sequence $(x_{i})_{i \in \mathbb{Z}}$ is stationary with respect to H if $D_2 H (x_{i - 1}, x_{i}) + D_1 (x_i, x_{i + 1}) = 0 \ \ \forall i \in \mathbb{Z}$.

\noindent \textbf{Resoult 1}: If $(x_{i})_{i \in \mathbb{Z}}$ is stationary with respect to H then $(x_{i}, y_{i})_{i \in \mathbb{Z}}$, $y_i = - D_{1} H(x_i, x_{i + 1} )$, is an orbit of $\bar{\varphi}$, i.e. $\bar{\varphi}(x_i, y_i) = (x_{i + 1}, y_{i + 1})$. Conversely, if $(x_{i}, y_{i})_{i \in \mathbb{Z}}$ is an orbit of $\bar{\varphi}$ then $(x_{i})_{i \in \mathbb{Z}}$  is stationary with respect to H.


\noindent \textbf{Observation}: Since the $(x_i)_{i \in \mathbb{Z}} \in \mathcal{M} (H)$ are stationary with respect to H every statement on minimal trajectories can be interpreted as a statement on certain types of orbits of the extended $\bar{\varphi}$. We can return to the original $\bar{\varphi}: \mathbb{R} \times [0, 1] \rightarrow \mathbb{R} \times [0, 1]$ in the following way: Let $f_0, f_1 \in 	\bar{G}_{+}$ be defined by $f_{0} (\xi) = f(\xi, 0)$, $f_{1} (\xi) = f(\xi, 1)$ and let $\alpha_0, \alpha_1$ be the rotation number of $f_0, f_1$. The interval $[ \alpha_0, \alpha_1 ]$ is called the \textbf{twist interval} of $\bar{\varphi}$. Later we will see that $\alpha_0 < \alpha_1$.


\noindent \textbf{Lemma}: Suppose $x \in \mathcal{M}_{\alpha}$ and $\alpha \in ( \alpha_0, \alpha_1)$, i.e. $\alpha$ is the interior of the twist interval of $\bar{\varphi}$. Then the orbit $(x_i, y_i)_{i \in \mathbb{Z}}$, $y_i = -D_1 H (x_i, x_{i + 1})$, is contained in 
$\mathbb{R} \times (0, 1)$. 


\noindent \textbf{Theorem 1}: For every irrational $\alpha \in ( \alpha_0, \alpha_1)$ there exists a $\varphi$-invariant set $\mathcal{M} \subseteq S^{1} \times (0, 1)$ with the following properties:
	\begin{itemize}
		\item[(a)] $M_{\alpha}$ is the graph of a Lipschitz function $\psi_{\alpha}: A_{\alpha} \rightarrow (0, 1)$ defined on a closet set $A_{\alpha} \subseteq S^{1}$.
		\item[(b)] $\varphi$ has rotation number $\alpha$  mod $(\mathbb{Z})$ on $M_{\alpha}$, i.e. there exits $h \in \mathcal{G}_{+}$ with $\alpha (h) \equiv \alpha$ mod($\mathbb{Z}$) such that $h(A_{\alpha}) = A_{\alpha}$ and
		
		$$
			\varphi ( \xi, \psi_{\alpha} (\xi) ) = ( h (\xi), \psi_{\alpha} (h(\xi) ) ) \ \ \forall \xi \in A_{\alpha} 		
		$$
		
		\item[(c)] The set $\mathcal{M}_{\alpha}^{\text{rec}}$ of recurrent points in $\mathcal{M}_{\alpha}$ projets either to a cantor set in $S^1$ or to all $S^{1}$. In the latter case $\mathcal{M}_{\alpha}$ is a $\varphi$-invariant Lipschitz curve winding once around the annulus $S^{1} \times [0, 1]$ and $\varphi | \mathcal{M}_{\alpha}$ is topologically conjugate to a rotation.
	\end{itemize}
 
\color{blue}
	\noindent \underline{\textbf{\textit{proof:}}} (we prove (a) - (c) for $\bar{\varphi}$ instead of $\varphi$; everything is invariant counting on $(\xi, \eta) \rightarrow (\xi + 1, \eta)$). Considering the previous lemma, we define:
	
	$$
		\bar{\mathcal{M}}_{\alpha} := \{
		(\xi, \eta) \in \mathbb{R} \times (0, 1) | \exists x \in \mathcal{M}_{\alpha} \ \text{ such that } \ \xi = x_0, \eta = - D_1 H(x_0, x_1)
		\}
	$$

\noindent Acording to \ref{eq1} and the resoult 1 the set $\bar{\mathcal{M}}_{\alpha}$ is $\varphi$-invariant. Acording to the resoutls INSERTAR AQUI LAS REFERENCIAS DE MIERDA MUHAHAHAHA the set $\bar{\mathcal{M}}_{\alpha}$ has one-to-one projection onto a closed set $\bar{A}_{\alpha} \subseteq \mathbb{R}$.

\noindent Considering REFERENCIA 3.19 MUHAHAHAHHA there exists a Lipschitz continuous $\bar{h} \in \bar{ \mathcal{G}_{+} }$ of a rotation number $\alpha$ such that $\bar{h} (x_0) = x_1 \ \ \forall x \in \mathcal{M}_{\alpha}$. Hence $\bar{\psi}_{\alpha} (\xi) := - D_{1} H (\xi, \bar{h} ( \xi ) )$ for $\epsilon \in \bar{A}_{\alpha}$ is Lipschitz as well.

\noindent This proves (a) and (b). Now (c) is a consequence of the properties of circle maps.


	\noindent $\spadesuit$
\color{black}
		
		
\noindent \textbf{Theorem 2}: If C is a $\varphi$-invariant curve winding once around $S^{1} \times [0, 1]$ and if $\varphi | C$ has rotation number $\alpha$ then $C \subseteq \mathcal{M}_{\alpha}$ and $C = \mathcal{M}_{\alpha}$ if $\alpha$ is irrational. 
		
		
%\color{blue}
%	\noindent \underline{\textbf{\textit{proof:}}} 		

%	\noindent $\spadesuit$
%\color{black}
		
\noindent \textbf{Lemma}: For every monotone twist map the twist interval has non-empty interior.

\noindent \textbf{Corollary}: For a standard diffeomorphism $\varphi: S^{1} \times \mathbb{R} \rightarrow S^{1} \times \mathbb{R}$ from (b) in the Theorem 1 choose the function s such that $s' (\xi_{0}) < -2$ for some $\xi_0 \in \mathbb{R}$. THen there is no $\varphi$-invariant curve winding around $S^1 \times \mathbb{R}$.


\color{blue}
	\noindent \underline{\textbf{\textit{proof:}}} 		

	\noindent We know from the Theorem 2 that the existence of a $\varphi$-invariant curve winding around $S^{1} \times \mathbb{R}$ implies the existence of an $x \in \mathcal{M} (H)$ with $x_0 = \xi_0$ where $H(\xi, \eta) = \frac{1}{2} ( \xi - \eta )^2 + S(\xi), \ \ S' = s$. Now we have:
		
	$$
		D_{22} H (x_{-1}, x_0, x_1 ) = D_{22} H( x_{-1}, x_0 ) + D_{11} H( x_0, x_1 ) = 2 + s' ( \xi_0 ) < 0	
	$$
	
	\noindent But this contradicts the minimality of the segment $(x_{-1}, x_0, x_1)$.
	
	\noindent $\spadesuit$
\color{black}

\noindent \textbf{Observation}: all this resoults are true (except the Lypschitz continuity in the Theorem 1) are also true if we relax the conditions of first definition (on a monotone twist map $\varphi$) in the following way:

\begin{itemize}
	\item[(a)] It is sufficient that $\varphi$ is a homeomorphism instead of being a $\mathcal{C}^{1}$-diffeomorphism.
	\item[(b)] Instead of $D_{2} f > 0$ it is suffices that $y \rightarrow f(\xi, y)$ is strictly increasing $\forall \xi \in \mathbb{R}$.
\end{itemize}

