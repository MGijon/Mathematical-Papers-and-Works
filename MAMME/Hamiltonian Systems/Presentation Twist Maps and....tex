\documentclass[12 pt]{beamer}
\usepackage[utf8]{inputenc}

\usepackage[]{amssymb}

\title{Twist Maps and Aubry-Mather Sets}
\author{\normalsize{Manuel Gijón Agudo}}
\date{}

\mode<presentation>{\usetheme{Madrid}}
\usecolortheme[ RGB={128,37,92} ]{structure}


\setbeamercolor{amarilloNota}{bg=yellow!50!white}
\setbeamercolor{azulNota}{bg=blue!15!white}
\setbeamercolor{naranjaNota}{bg=orange!15!white}


\begin{document}

\begin{frame}
    \begin{center}

        Universidad Politécnica de Cataluña
        
        Facultad de Matemáticas y Estadística
        
        MAMMEE
        
        \scriptsize{Hamiltonian Systems}
        
        \maketitle    
        
        \small{}
    \end{center}
  
\end{frame}


\begin{frame}{0}
    \frametitle{Index}     
    \tableofcontents
\end{frame}

%% ============ %%
%% INTRODUCTION %%
%% ============ %%

\section{Introduction}

\begin{frame}{1}
    \frametitle{}
   
  	\begin{itemize}
  		\item[-] We will be working with \textbf{area-preserving monotone twist maps} preserving the boundary components
			$$
				\varphi: S^{1} \times [0, 1] \longrightarrow S^{1} \times [0, 1]
			$$
		\item[-] The \textbf{Mather sets} will be a particular $\varphi$-invariant subsets of the cylinder.
		\item[-] We want to find $\varphi$-invariant closed curves which separates $S^{1} \times \{ 0 \}$ and $S^{1} \times \{ 1 \}$. They are related with the stability of the system $\{ \varphi^{n} \}_{n \in \mathbb{N}}$.	
  	\end{itemize}

\end{frame}

\begin{frame}{2}
    \frametitle{Curves and stability of $\{ \varphi^{n} \}_{n \in \mathbb{N}}$}
   
\end{frame}

\begin{frame}{3}
    \frametitle{Relationship with KAM theory}
   	\begin{itemize}
   		\item[-] KAM-Theory for this kind of maps shows that for this kind of maps which are sufficiently $\mathcal{C}^{k}$-close to an integrable one, then many of the invariant curves persists. The invariant curves are destroyed when we go too far away from the integrable situation and the Mather sets, $M_{\alpha}$ re the most important remmants of the invariant curves of irrational rotation number $\alpha$.
   		\item[-] The Aubry-Mather theory appears to explain what happend with perturbed systems and its main contribution is to explain what happend with the orbits in the cases that are not covered by KAM theory.
   	\end{itemize}
\end{frame}

%% ====  %%
%% MAPS %%
%% ==== %%

\section{Working with maps}

\begin{frame}{4}
	\frametitle{}
	\begin{itemize}
		\item[-] \textbf{Trajectory}: $x = (x_{i})_{i \in \mathbb{Z}} \in \mathbb{R}^{\mathbb{Z}}$.
		\item[-] \textbf{Convergence} is the obvious notion (with the product topology) for each $i$.
		\item[-] Given a function $H:\mathbb{R}^{2} \rightarrow \mathbb{R}$ we can extend it to trajectories (or finite segments) by:
			$$
				H(x_j, \cdots, x_k) = \sum_{i = j}^{k - 1} H(x_{i}, x_{i + 1})
			$$
	\end{itemize}
\end{frame}
 
%% ======= %%
%% RESULTS %%
%% ======= %%

\section{Main results}

\subsection{M.T. with irrational rotation number}


\subsection{M.T. with rational rotation number}


%% =========== %%
%% BIBLIOGRAPHY %%
%% =========== %%

\begin{frame}{}
    \frametitle{Bibliography}
    
    
    \begin{thebibliography}{ABC9999}

    \bibitem[R1]{}
    V. Banget,
    \textit{Mather sets for twist maps and geodesic on tori}.
	    
	\bibitem[R2]{}
    Y. Katznelson, D.S. Ornstein, 
    \textit{Twist Maps and Aubry-Mather Sets}.
    
    \bibitem[R3]{}
    A. Katok, B. Hasselblatt,
    \textit{Introduction to the Modern Theory of Dynamical System}
    
    \end{thebibliography}
        
\end{frame}

\end{document}
