\documentclass[12 pt]{beamer}
\usepackage[utf8]{inputenc}

\usepackage[]{amssymb}

\usepackage{multimedia}
\usepackage{media9}

\title{Twist Maps and Aubry-Mather Sets}
\author{\normalsize{Manuel Gijón Agudo}}
%\date{10 July 2017}
\date{}

\mode<presentation>{\usetheme{Madrid}}
%\mode<presentation>{\usetheme[secheader]{Boadilla}}
\usecolortheme[ RGB={128,37,92} ]{structure}

\begin{document}

\begin{frame}[plain]
    \begin{center}

        Universidad Politécnica de Cataluña
        
        Facultad de Matemáticas y Estadística
        
        MAMMEE
        
        \scriptsize{Hamiltonian Systems}
        
        \maketitle    
        \\*
        
        \small{}
    \end{center}
    
    
\end{frame}

% DEFINICIONES DE COLORES
\setbeamercolor{amarilloNota}{bg=yellow!50!white}
\setbeamercolor{azulNota}{bg=blue!15!white}
\setbeamercolor{naranjaNota}{bg=orange!15!white}


% DIAPOSITIVA 0
\begin{frame}{0}
    \frametitle{Index}
     
    \tableofcontents  
\end{frame}



\section{Working with maps}

% DIAPOSITIVA 1
\begin{frame}{1}
    \frametitle{}
    
    
\end{frame}

% DIAPOSITIVA 2
\begin{frame}{2}
    \frametitle{I}
   
\end{frame}

% DIAPOSITIVA 3
\begin{frame}{3}
    \frametitle{Thickness of a graph $\mathcal{G}$}
  
    \begin{block}{Theorical Thickness $\theta (\mathcal{G})$}
    Minimum number of planar graphs into a which a graph can be descomposed.
    \end{block}
    
    \pause
    
    \begin{block}{Geometric Thickness $\bar{\theta} (\mathcal{G})$}
    Smallest value of $k$ such that we can assign a planar point locations to the vertices of $\mathcal{G}$, represent each edge of $\mathcal{G}$ as a line segment, and assign each edge to one of $k$ layers so that no two edges on the same layer cross.
    \end{block}

\end{frame}

% DIAPOSITIVA 4
\begin{frame}{4}
    \frametitle{Thickness of a graph}
  
    \begin{alertblock}{Key difference}
    Geometric thickness requires that the vertex placements be consistent at all layers and that straight-line adges be used, whereas graph-theorical thickness imposes no consistency requirement between layers.
    \end{alertblock}
\end{frame}

%%%%%%%%%%%%%%%%%%%%%%%%%%%%%%%%%%%%%%%%%%%%%%%%%%%%%%%%%%%%%%%%%%%%%%%%%%%%
\section{Results}

    \subsection{Good news}

% DIAPOSITIVA 5.a
\begin{frame}{5}
    \frametitle{Graph-theorical thickness for all complete graphs}
    


\end{frame}

% DIAPOSITIVA 5.b
\begin{frame}{5}
    \frametitle{Graph-theorical thickness for all complete graphs}
    
   
\end{frame}

    \subsection{Upper and lower bounds}

% DIAPOSITIVA 6.a
\begin{frame}{}
    \frametitle{Upper Bounds}
    
    \begin{block}{\textbf{Theorem 1}}
    $$\bar{\theta} (k_{n}) \leq \lceil n/4 \rceil $$
    \end{block}
    
    \pause
    
    \textbf{Roadmap to the proof:}
    
    \begin{itemize}
        \item Assume that $n$ is multiple of four, $n = 2k$ with ($k$ even), show that $n$ vertices can be arranged in two ``rings'' of $k$ vertices, so $K_{n}$ can be embedded using $k/2$ layers and with no edges on the same layer crossing.
    \end{itemize}
\end{frame}

% DIAPOSITIVA 6.b
\begin{frame}{}
    \frametitle{Upper Bounds}

    \textbf{Roadmap to the proof:}
    
    \begin{itemize}
        \item Use the vertices of the inner ring to form a regular $k$-gon and considerer the opposite vertices con create a zigzag path. This path has exactly one diagonal connecting diametrically opposite points. 
        
        \item By continuity, we can replace by a suitably chosen common end points the infinite end points of a collection of parellel rays. Thus forming an outer ring of $k$ vertices.
        
        \item The figure can be perturbed by moving slightly the inner ring. None of the diagonals of the polygon comprinsing the outher ring intersect the polygon comprising the inner ring.
        
        \item It's straighforward to verify that this is indeed a descomposition of the edges of $k_n$ into $k/2 = n/4$ layers.
    \end{itemize}
    
    \begin{flushright}
    $\blacksquare$
    \end{flushright}
\end{frame}


% DIAPOSITIVA 7
\begin{frame}{}
    \frametitle{Lower Bounds}
    
    \begin{block}{\textbf{Theorem 2}} 
    For all $n \geq 1$
    $$\bar{\theta} (k_{n}) \geq \max_{1 \leq x \leq n/2} \frac{\binom{n}{2} - 2\binom{x}{2} - 3}{3n - 2x - 7}$$
    \end{block}
    
    \pause
    
    \begin{exampleblock}{In particular, for $n \geq 12$}
    $$
    \bar{\theta} (k_{n}) \geq \left \lceil \frac{3 - \sqrt{7}}{2}n + 0.342 \right \rceil  \geq \left \lceil \frac{n}{5.646} + 0.342 \right  \rceil 
    $$
    \end{exampleblock}
\end{frame}

    \subsection{$K_{15}$}
    
% DIAPOSITIVA 8.a
\begin{frame}{}
    \frametitle{}
    
    \begin{block}{\textbf{Theorem 3}}
    $$\bar{\theta} (k_{15}) = 4 $$
    \end{block}
 
\end{frame} 

% DIAPOSITIVA 8.b
\begin{frame}{}
    \frametitle{Geometric Thickness of $K_{15}$}
    
    \begin{block}{\textbf{Theorem 3}}
    $$\bar{\theta} (k_{15}) = 4 $$
    \end{block}

    \textbf{Proof:}
    We divide the problem in $3$ cases.
    
    \begin{itemize}
    \item \textbf{Case II:} $4$ points in the convex hull. 
        \begin{itemize}
            \item Let $\mathcal{A, B, C, D}$ the convex hull vertices.
            \item Assume triangle $\mathcal{DAB}$ has at least one point of $\mathcal{S}$ in his interior (if not switch $\mathcal{A}$ and $\mathcal{C}$ and let $\mathcal{A}_{1}$ be the point inside furthest from line $\mathcal{DB}$).
            \item By Lemma 2, the edge $\mathcal{AA}_{1}$ must appear in every triangulation of $\mathcal{S}$.
            \item Since every triangulation has $38$ edges three triangulations can account at most $104$ edges.
        \end{itemize}
    \end{itemize}
\end{frame} 

% DIAPOSITIVA 8.c
\begin{frame}{}
    \frametitle{Geometric Thickness of $K_{15}$}
    
    \begin{block}{\textbf{Theorem 3}}
    $$\bar{\theta} (k_{15}) = 4 $$
    \end{block}

    \textbf{Proof:}
    We divide the problem in $3$ cases.
    
    \begin{itemize}
    \item \textbf{Case III:} $5$ or more points in the convex hull. 
        \begin{itemize}
            \item Let $h$ be the number of points in the convex hull.
            \item A triangulation of $\mathcal{S}$ will have $42 - h$ edges, and all hull edges must be in each triangulation.
            \item The total number of edges in three triangulatinos is at most $3(42 -  2h) + h = 126 - -5h$, that is at most $101$ for $h \geq 5$.
        \end{itemize}
    \end{itemize}
    
    \begin{flushright}
    $\blacksquare$
    \end{flushright}
\end{frame} 

    \subsection{Bipartite graphs}
    
% DIAPOSITIVA 9
\begin{frame}{}
    \frametitle{Geometric Thickness of Complete Bipartite Graphs}
    
    \begin{block}{\textbf{Theorem 4}}
    For the complete bipartite graph $K_{a, b}$
    $$\left \lceil \frac{ab}{2a + 2b - 4} \right \rceil \leq \theta (K_{a,b}) \leq \bar{\theta} (k_{a,b}) \leq \left \lceil \frac{\min(a, b)}{2}  \right \rceil$$
    \end{block}
    
    \pause
    
    \textbf{Proof:}
    \begin{itemize}
    \item First inequality: from Euler's formula since a bipartite graph which is planar with $a + b$ vertices can have at most $2a + 2b -4$ edges
    \item Second inequality: assume that $a \leq b$ and $a$ is even. Draw $b$ vertices in a horizontal line, with $a/2$ red vertives above the line and $a/2$ vertices below.
    Each layer consists of all edges connecting the blue vertices with one red vertex from above the line and one red vertex from below.
    \end{itemize}
    
    \begin{flushright}
    $\blacksquare$
    \end{flushright}
\end{frame}

% DIAPOSITIVA 10
\begin{frame}{}
    \frametitle{Geometric Thickness of Complete Bipartite Graphs}
    
    \begin{block}{\textbf{Corollary 1:}}
    For any integer $b$, $\bar{\theta} (k_{a,b}) = \theta (k_{a,b})$ provided:
    $$ a >
    \begin{cases}
    \frac{(b-2)^2}{2}  & \text{if b is even}\\
    (b-1)(b-2)  & \text{if b is odd}
    \end{cases}
    $$
    \end{block}
    
    \pause
    
    \textbf{Proof:}
    If $a > b$, the leftmost and rightmost quantities in the expresion of the last theorem will be equal provided $ab/(2a + 2b -4) > (b - 2)/2$ if $b$ is even, or provided  $ab/(2a + 2b -4) > (b - 1)/2$ f $b$ is odd. By simplifing this inequality holds.
    
    \begin{flushright}
    $\blacksquare$
    \end{flushright}
\end{frame}



    \subsection{To sum up}
    
% DIAPOSITIVA 12.a
\begin{frame}{}
    \frametitle{The first 100 values of  $\bar{\theta}$}
 
\end{frame}




% DIAPOSITIVA FINAL
\begin{frame}{}
    \frametitle{Bibliography}
    
    
    \begin{thebibliography}{ABC9999}

    \bibitem[Ben]{Bens}
    D.Dillencourt, D. Eppstein and D.\,S. Hirschberg,
    \textit{Geometric thickness of complete graphs}, 
        J. Graph Algorithmss and Applications 4 (3), 5-17, 2000.
        %http://www.maths.abdn.ac.uk/∼bensondj/
	    
	\bibitem[PTM]{PTM}
    P. Mutzel, T. Odenthal, M. Scharbrodt, 
    \textit{The thickness og graphs: a survey}, 
        Graphs and Combinatorics, March 1998, Volume 14, Issue 1, pp 59–73.
    \end{thebibliography}
        
\end{frame}

\end{document}